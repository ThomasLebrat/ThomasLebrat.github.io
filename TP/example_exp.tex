% Created 2023-01-22 dim. 22:34
% Intended LaTeX compiler: pdflatex
\documentclass[11pt]{article}
\usepackage[utf8]{inputenc}
\usepackage{lmodern}
\usepackage[T1]{fontenc}
\usepackage[top=1in, bottom=1.in, left=1in, right=1in]{geometry}
\usepackage{graphicx}
\usepackage{longtable}
\usepackage{float}
\usepackage{wrapfig}
\usepackage{rotating}
\usepackage[normalem]{ulem}
\usepackage{amsmath}
\usepackage{textcomp}
\usepackage{marvosym}
\usepackage{wasysym}
\usepackage{amssymb}
\usepackage{amsmath}
\usepackage[theorems, skins]{tcolorbox}
\usepackage[version=3]{mhchem}
\usepackage[numbers,super,sort&compress]{natbib}
\usepackage{natmove}
\usepackage{url}
\usepackage[cache=false]{minted}
\usepackage[strings]{underscore}
\usepackage[linktocpage,pdfstartview=FitH,colorlinks,
linkcolor=blue,anchorcolor=blue,
citecolor=blue,filecolor=blue,menucolor=blue,urlcolor=blue]{hyperref}
\usepackage{attachfile}
\usepackage{setspace}
\author{John Kitchin}
\date{\today}
\title{Example of org to ipynb}
\begin{document}

\tableofcontents


\section{Introduction}
\label{sec:org900561c}

I wrote this module to allow me to write lecture notes in org-mode, but export them to ipython notebooks for my students. It also makes it easier to share my work with people who don't use org-mode (although it is a one-way conversion of org to ipynb).

To use this, you have to require the ox-ipynb library. Then you can export this file with the key sequence \texttt{C-c C-e n o} to create the ipynb and open it.

In the following sections I will demonstrate a few features.

\subsection{Org-mode markup}
\label{sec:orgc699949}
You should get all the regular markup:

\begin{itemize}
\item \textbf{bold text}
\item \emph{italics}
\item \uline{underlined}
\item \sout{strikethrough}
\item \texttt{verbatim}
\item \texttt{code}
\item superscripts, e.g. H\^{}+
\item subscripts, e.g. CH\textsubscript{4}
\item hyperlinks: \url{http://github.com/jkitchin/ox-ipynb}
\item internal links do not work: \hyperref[sec:orgc699949]{Org-mode markup}
\end{itemize}

\subsection{Embed equations}
\label{sec:orgd509c05}

You can use \LaTeX{} equations like: \(\int_0^1 \sin x dx\).

\subsection{Embed code and results}
\label{sec:orgcc8d5e7}

Note that code blocks must be either "ipython" or "R", and the first one determines the language used for all of the code blocks.

\begin{minted}[frame=lines,fontsize=\scriptsize,linenos]{ipython}
import numpy as np
from scipy.integrate import quad

print(quad(lambda x: np.sin(x), 0, 1))
\end{minted}

(0.45969769413186023, 5.103669643922839e-15)


Here we force a new cell to be created with the \texttt{ipynb-newcell} directive.
\#+ipynb-newcell

This text should be in its own cell.

\subsection{Embed tables}
\label{sec:orgd06a1ee}

You can create tables and they render nicely in the notebook. Note that you probably should not use the tables as data sources to code blocks because the ipython notebook does not support that.

\begin{table}[htbp]
\caption{A table of numbers.}
\centering
\begin{tabular}{rr}
x & y\\
\hline
1 & 1\\
2 & 4\\
3 & 9\\
\end{tabular}
\end{table}


\subsection{Use ipython magic to get inline figures}
\label{sec:org05718ab}

\begin{minted}[frame=lines,fontsize=\scriptsize,linenos]{ipython}
%matplotlib inline
import matplotlib.pyplot as plt

z = np.linspace(0, 20 * np.pi, 500)
x = np.sin(z) * np.exp(-0.1 * z)
y = np.cos(z) * np.exp(-0.1 * z)
plt.plot(x, y)
\end{minted}



[<matplotlib.lines.Line2D at 0x11a2ef6a0>]
<matplotlib.figure.Figure at 0x11a22b588>
\begin{center}
\includegraphics[width=.9\linewidth]{ipython-inline-images/ob-ipython-3463501494cf7e34b4ab1216c3e45e82.png}
\end{center}

\subsection{Store metadata in the ipynb}
\label{sec:org002206d}

The directive \texttt{ox-ipynb-keyword-metadata} lists file variables that should be saved as metadata in the ipynb.

We can set values as file variables like this: 


\begin{center}
\includegraphics[width=.9\linewidth]{./screenshots/date-05-08-2017-time-08-55-41.png}
\end{center}

On export, you can see these are stored in the metadata as:



\begin{center}
\includegraphics[width=.9\linewidth]{./screenshots/date-05-08-2017-time-08-56-35.png}
\end{center}

These will show as bullets at the top of the ipynb. These are machine readable, so you can access them with any tool that can read json. You can use this to store author names, document id's, etc.

\begin{minted}[frame=lines,fontsize=\scriptsize,linenos]{ipython}
import json

with open("example.ipynb") as f:
    d = json.loads(f.read())

print(d['metadata']['org'])
\end{minted}

\{'KEY1': 'value1', 'KEY2': 'value2'\}

\subsection{Use HTML for fancy markup}
\label{sec:org3784083}

You can use html tags directly in the org file. Here are two examples.

 <div class="alert alert-warning">
You can use an alert to highlight something you want to stand out.
 </div>

<font color="blue">This text will show up as blue.</font>
\end{document}
