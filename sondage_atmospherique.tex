% Created 2023-01-20 ven. 14:04
% Intended LaTeX compiler: pdflatex
\documentclass[11pt]{article}
\usepackage[utf8]{inputenc}
\usepackage{lmodern}
\usepackage[T1]{fontenc}
\usepackage[top=1in, bottom=1.in, left=1in, right=1in]{geometry}
\usepackage{graphicx}
\usepackage{longtable}
\usepackage{float}
\usepackage{wrapfig}
\usepackage{rotating}
\usepackage[normalem]{ulem}
\usepackage{amsmath}
\usepackage{textcomp}
\usepackage{marvosym}
\usepackage{wasysym}
\usepackage{amssymb}
\usepackage{amsmath}
\usepackage[theorems, skins]{tcolorbox}
\usepackage[version=3]{mhchem}
\usepackage[numbers,super,sort&compress]{natbib}
\usepackage{natmove}
\usepackage{url}
\usepackage[cache=false]{minted}
\usepackage[strings]{underscore}
\usepackage[linktocpage,pdfstartview=FitH,colorlinks,
linkcolor=blue,anchorcolor=blue,
citecolor=blue,filecolor=blue,menucolor=blue,urlcolor=blue]{hyperref}
\usepackage{attachfile}
\usepackage{setspace}
\author{maint}
\date{\today}
\title{}
\begin{document}

\tableofcontents

\section{test}
\label{sec:orgdf9ec41}

\begin{table}[htbp]
\label{tabtest}
\centering
\begin{tabular}{rr}
toto & tata\\
1 & 2\\
3.14 & 6.28\\
\end{tabular}
\end{table}


\begin{minted}[frame=lines,fontsize=\scriptsize,linenos]{python}
print(tabtest)
\end{minted}


\subsection{réglages}
\label{sec:org857ee25}


\begin{table}[htbp]
\label{mydata}
\centering
\begin{tabular}{lr}
Drug & Patients\\
\hline
X & 232\\
Y & 3?1\\
Z & 123\\
\end{tabular}
\end{table}

\begin{minted}[frame=lines,fontsize=\scriptsize,linenos]{python}
print(tbl)
\end{minted}



\section{TP}
\label{sec:org316f71a}


\subsection{saisie du tableau}
\label{sec:orgedb7f27}
\begin{table}[htbp]
\label{temp}
\centering
\begin{tabular}{rr}
z(km) & T(°C)\\
\hline
0 & 15\\
5 & -18\\
10 & -49\\
12 & -56\\
20 & -56\\
-51 & -46\\
30 & -37\\
35 & -22\\
40 & -8\\
45 & -2\\
48 & -2\\
52 & -7\\
55 & -17\\
60 & -33\\
65 & -54\\
70 & -65\\
75 & -79\\
80 & -86\\
84 & -86\\
92 & -86\\
95 & -81\\
100 & -72\\
\end{tabular}
\end{table}


\begin{minted}[frame=lines,fontsize=\scriptsize,linenos]{python}
print(tbl[:][1])
\end{minted}

[5, -18]


\subsubsection{parenthese pandas}
\label{sec:org2c8dd3b}

\begin{minted}[frame=lines,fontsize=\scriptsize,linenos]{python}
import pandas as pd
D = pd.DataFrame(tbl).iloc[:, :]
print(D)
\end{minted}

      0   1
0     0  15
1     5 -18
2    10 -49
3    12 -56
4    20 -56
5   -51 -46
6    30 -37
7    35 -22
8    40  -8
9    45  -2
10   48  -2
11   52  -7
12   55 -17
13   60 -33
14   65 -54
15   70 -65
16   75 -79
17   80 -86
18   84 -86
19   92 -86
20   95 -81
21  100 -72

\begin{minted}[frame=lines,fontsize=\scriptsize,linenos]{python}
import pandas as pd
D = pd.DataFrame(tbl).iloc[:, 1:3]
print(D)
\end{minted}

     1
0   15
1  -18
2  -49
3  -56
4  -56
5  -46
6  -37
7  -22
8   -8
9   -2
10  -2
11  -7
12 -17
13 -33
14 -54
15 -65
16 -79
17 -86
18 -86
19 -86
20 -81
21 -72


\begin{minted}[frame=lines,fontsize=\scriptsize,linenos]{python}
#Altitude
print(D)
\end{minted}

     1
0   15
1  -18
2  -49
3  -56
4  -56
5  -46
6  -37
7  -22
8   -8
9   -2
10  -2
11  -7
12 -17
13 -33
14 -54
15 -65
16 -79
17 -86
18 -86
19 -86
20 -81
21 -72


\subsection{values}
\label{sec:org7030535}
\begin{minted}[frame=lines,fontsize=\scriptsize,linenos]{python}
M = 29.0e-3
R = 8.31
P0 = 1.0e5
g0 = 9.8
RT = 6.4e3
\end{minted}

\begin{minted}[frame=lines,fontsize=\scriptsize,linenos]{python}
import numpy as np
import matplotlib
import matplotlib.pyplot as plt
\end{minted}

\begin{minted}[frame=lines,fontsize=\scriptsize,linenos]{python}
zexp = np.array([0.0, 5.0, 10.0, 12.0, 20.0, 25.0, 30.0, 35.0, 40.0, 45.0, 48.0, 52.0, 55.0, 60.0, 65.0, 70.0, 75.0, 80.0, 84.0, 92.0, 95.0, 100.0])

Texp = np.array([15.0, -18.0, -49.0, -56.0, -56.0, -51.0, -46.0, -37.0, -22.0, -8.0, -2.0, -2.0, -7.0, -17.0, -33.0, -54.0, -65.0, -79.0, -86.0, -86.0, -81.0, -72.0])

# print(len(zexp))
# print(len(Texp))
# print(zexp)
# print(Texp)
\end{minted}


\subsection{interpolation}
\label{sec:org7720918}
On considère deux points de mesure \(i\) et \(i+1\), on a la relation $$T_k = a z_k +b $$ avec \(a\) et \(b\) indéterminés. Evcrivons la relation de la température en \(k=i\) et \(k=i+1\)

\begin{minted}[frame=lines,fontsize=\scriptsize,linenos]{python}
def T(z,unite):
    z_km = z / 1000 #conversion
    alpha = 1 # facteur de conversion
    
    if unite == 'C':
        alpha = 0
        
    i = 0
    while z_km > zexp[i+1]: # recherche de l'indice i
        i = i + 1
        
    rate =  ( Texp[i+1] - Texp[i] ) / ( zexp[i+1] - zexp[i] )
    temperature = alpha*273 + Texp[i] + rate * (z_km - zexp[i])
    return temperature

\end{minted}


\subsection{gravity}
\label{sec:orge117953}
\begin{minted}[frame=lines,fontsize=\scriptsize,linenos]{python}
def g(z):
    return g0 * RT**2 / (RT + z) **2
\end{minted}


\subsection{temperature}
\label{sec:orgad2f3cd}
\begin{minted}[frame=lines,fontsize=\scriptsize,linenos]{python}
N = 1000
zmax = 100.0e3
dz = zmax / (N-1)
print(N, zmax, dz)
zatm = np.array([ k * dz for k in range(N) ])
Tatm = np.array([ T(zatm[k], 'C') for k in range(N) ])
TatmK = np.array([ T(zatm[k], 'K') for k in range(N) ])
gatm = np.array([ g(zatm[k]) for k in range(N)])
\end{minted}

1000 100000.0 100.10010010010011

\begin{minted}[frame=lines,fontsize=\scriptsize,linenos]{python}
fig, ax = plt.subplots()
ax.plot( TatmK,zatm)
ax.plot( Tatm,zatm)
plt.savefig("ffffffffff")
\end{minted}


\begin{center}
\includegraphics[width=.9\linewidth]{./.ob-jupyter/ddb61d97953cb9f84965f5198e45a9d658967ac2.png}
\end{center}


\subsection{pressure}
\label{sec:org6b0ed68}
calcul du champ de pression par la méthode d'Euler
\begin{minted}[frame=lines,fontsize=\scriptsize,linenos]{python}
Patm = [P0]
gatm = [g0]
deltap = 0
gradient = 0
for k in range(N-1):
    gradient = - M * g( zatm[k] ) / (R * TatmK[k] )
    deltap = gradient * dz
    Patm.append( Patm[k] + deltap )
#    gatm.append( gatm[k] )
Patm = np.array(Patm)
print(M,R,P0,g0,RT)
\end{minted}

0.029 8.31 100000.0 9.8 6400.0

\begin{minted}[frame=lines,fontsize=\scriptsize,linenos]{python}
plt.plot(Patm,zatm)
\end{minted}

\begin{center}
\includegraphics[width=.9\linewidth]{./.ob-jupyter/6ab5d538c5d7f7e5a4224a6939d11097f5fb56d1.png}
\end{center}


\subsection{masse d'air}
\label{sec:orged4df8d}

calcul de la masse d'air par la méthode des rectangles
situé entre deux sphères d'altitude z et z+dz

\begin{minted}[frame=lines,fontsize=\scriptsize,linenos]{python}
def masse_atm(z):
    masse = 0
    k = 0
    while zatm[k] < z:
        dm = dz * 4 * np.pi * (RT + z)**2 * M * Patm[k] / (R* T(zatm[k],'K'))
        masse = masse + dm
        k = k+1
    return masse
\end{minted}

\begin{minted}[frame=lines,fontsize=\scriptsize,linenos]{python}

mtot = masse_atm(100e3)
print(mtot)

\end{minted}

2.201395425007424e+16

\subsection{next}
\label{sec:org29b6aa9}
\end{document}
